\documentclass{article}
\usepackage{listings}% http://ctan.org/pkg/listings
\usepackage{amsmath,amsthm,amssymb,amsfonts}
\usepackage{changepage,lipsum,titlesec}% http://ctan.org/pkg/{changepage,lipsum,titlesec}
\usepackage{enumitem}
\usepackage{underscore} % Allows literal underscores
\usepackage{fixltx2e} % Allows \textsubscript{}
\usepackage{graphicx}
\usepackage{titling}
\usepackage{color}
\usepackage{pdfpages}
\usepackage{tikz} % For diagrams
\usepackage{hyperref}
\usepackage{fancyhdr} % For headers and footers
\usepackage{geometry}
\usepackage{mathtools}
\usepackage{mathrsfs}

 \geometry{
 a4paper, % Change this if you intend to print on a different paper size, such as letter paper.
 total={210mm,297mm},
 left=20mm,
 right=20mm,
 top=30mm,
 bottom=30mm,
 }
\pagestyle{fancy}

%Settings
\newcommand{\courseName}{MAFS5130 Time Series Analysis} % Insert course name here
\title{\courseName}
\author{Thompson Hu}
\date{} % If you'd like to have a date, place it here

% Headers and footers. See documentation for 'fancyhdr' package for more information.
\lhead{\courseName} % Left header
\chead{} % Center header
\rhead{} % Right header
\lfoot{} % Left footer
\cfoot{\thepage} % Center footer
\rfoot{} % Right footer

% Lecture block settings.
\lstset{
  basicstyle= \normalfont,
  mathescape=true, % Allows for adding equations.
  escapechar={~}, % You may wish to change this symbol, if you use ~ in your notes a lot.
  breaklines=true, % Automatic text wrapping.
  columns=fullflexible
}

\renewcommand{\headrulewidth}{0.4pt}
\renewcommand{\footrulewidth}{0.4pt}
\renewcommand*\rmdefault{ppl} %Set font

\begin{document}
\maketitle
\thispagestyle{fancy}

\section{Basic concepts}
The foundation of time series analysis is stationarity.

\subsection{Strictly stationary process}
For $n$ and $t_1,t_2,\cdots,t_n$, if
\begin{equation*}
\left(X_{t_{1}}, \cdots, X_{t_{n}}\right) \stackrel{d i s t}{=}\left(X_{t_{1}+k}, \cdots, X_{t_{n}+k}\right), \forall k
\end{equation*}
we say $\{X_t\}$ is strictly stationary process.

\subsection{Weakly stationary process}
If $\{X_t\}$ satisfied that
\begin{itemize}
	\item $E\left(X_{t}^{2}\right)<\infty$;
	\item $E(X_t) = \mu$, which is constant;
	\item $\operatorname{Cov}\left(X_{t}, X_{t+k}\right)$ is not related to $t$ for $k$.
\end{itemize}
we say $X_t$ is weakly stationary process.

\subsection{Autocorrelation function (ACF)}
Supposed $X_t$ is weakly stationary sequence, then
\begin{equation}\label{acf1}
\rho\left(X_{t-k}, X_{t}\right)=\frac{\operatorname{Cov}\left(X_{t-k}, X_{t}\right)}{\sqrt{\operatorname{Var}\left(X_{t-k}\right) \operatorname{Var}\left(X_{t}\right)}}=\frac{\gamma_{k}}{\sqrt{\gamma_{0} \gamma_{0}}}=\frac{\gamma_{k}}{\gamma_{0}}, k=0,1, \ldots, \forall t
\end{equation}
We say $\{\rho_k, k = 0, 1, \dots\}$ is autocorrelation function (ACF) of $\{X_t\}$ and $\rho_0 = 1$.\\

\noindent If weakly stationary sequence $\{X_t\}$ satisfy that $\rho_k = 0, k = 1, 2, \dots$, then $\{X_t\}$ is \textbf{white noise} sequence.\\

\noindent From the definition \eqref{acf1}, we know
\begin{equation*}
\rho_k = \frac{\operatorname{Cov}\left(X_{t-k}, X_{t}\right)}{\operatorname{Var}\left(X_{t}\right)} = \frac{\operatorname{Cov}\left(X_{t}, X_{t-k}\right)}{\operatorname{Var}\left(X_{t}\right)} = \rho_{-k}
\end{equation*}

\subsection{Partial ACF (PACF)}
Supposed $X_t$ is a stationary sequence, the conditional correlation
\begin{equation*}
\operatorname{Corr}\left(X_{t}, X_{t+k} | X_{t+1}, \cdots, X_{t+k-1}\right) = \frac{\operatorname{Cov}\left[\left(X_{t}-\hat{X}_{t}\right)\left(X_{t+k}-\hat{X}_{t+k}\right)\right]}{\sqrt{\operatorname{Var}\left(X_{t}-\hat{X}_{t}\right) \operatorname{Var}\left(X_{t+k}-\hat{X}_{t+k}\right)}}
\end{equation*}
is called the PACF of $X_t$ and $X_{t+k}$, denoted by $\phi_{kk}$, where $\hat{X}_{t}=E\left(X_{t} | X_{t+1}, \cdots, X_{t+k-1}\right).$\\

\noindent From the formula : $\phi_{11} = \rho_1$ and
\begin{equation*}
\phi_{kk} = \frac{
\left| \begin{array}{cccccc}{1} & {\rho_{1}} & {\rho_{2}} & {\cdots} & {\rho_{k-2}} & {\rho_{1}} \\ {\rho_{1}} & {1} & {\rho_{1}} & {\cdots} & {\rho_{k-3}} & {\rho_{2}} \\ { } & { } & {\cdots} & { } & { } \\ { } & { } & {\cdots} \\ { } & { } & {\cdots} \\ {\rho_{k-1}} & {\rho_{k-2}} & {\rho_{k-3}} & {\cdots} & {\rho_{1}} & {\rho_{k}}\end{array}\right|}{
\left| \begin{array}{cccccc}{1} & {\rho_{1}} & {\rho_{2}} & {\cdots} & {\rho_{k-2}} & {\rho_{k-1}} \\ {\rho_{1}} & {1} & {\rho_{1}} & {\cdots} & {\rho_{k-3}} & {\rho_{k-2}} \\ { } & { } & {\cdots} & { } & { } \\ { } & { } & {\cdots} \\ {\rho_{k-1}} & {\rho_{k-2}} & {\rho_{k-3}} & {\cdots} & {\rho_{1}} & {1}\end{array}\right|
}
\end{equation*}
we know the key feature is that PACF cuts off at lag p for an AR(p) model.

\subsection{Ljung-Box test}
Box and Pierce presented portmanteau statistic
\begin{equation}\label{boxtest1}
Q_{*}(m)=T \sum_{j=1}^{m} \hat{\rho}_{j}^{2}
\end{equation}
in 1970 to test
$$
H_{0} : \rho_{1}=\cdots=\rho_{m}=0 \longleftrightarrow H_{a} : \rho_i \neq 0\ \text{for some } i 
$$

\noindent In 1978, Ljung and Box improved such method by changing previous statistic \eqref{boxtest1} as
\begin{equation*}
Q(m)=T(T+2) \sum_{j=1}^{m} \frac{\hat{\rho}_{j}^{2}}{T-j}
\end{equation*}

\noindent If $Q(m) > \chi_m^2(\alpha)$ or p-value is less than $\alpha$, then we reject $H_0$.

\subsection{Linear time series}
$X_t$ is linear if $X_t$ can be written as
\begin{equation*}
X_{t}=\mu+\sum_{i=0}^{\infty} \psi_{i} Z_{t-i}
\end{equation*}
where $\mu$ is a constant, $\psi_0 = 1$ and $\{Z_t\}$ is a sequence of white noises.

\subsection{Information criterion}
\begin{itemize}
	\item Akaike information criterion
	\begin{equation*}
	\text{AIC}(k) = \ln \left(\hat{\sigma}_{k}^{2}\right)+\frac{2 k}{T}
	\end{equation*}
	for an AR(k) model, where $\sigma_k^2$ is the MLE of residual variance.
	
	\item BIC criterion
	\begin{equation*}
	\text{BIC}(k) = \ln \left(\hat{\sigma}_{k}^{2}\right)+\frac{k \ln (T)}{T}
	\end{equation*}
\end{itemize}


\section{ARMA model}
\subsection{Auto Regression (AR) model}
\subsubsection{AR(1) model}
The form of AR(1) can be written as:
\begin{equation}\label{AR1}
X_{t}=\phi_{0}+\phi_{1} X_{t-1}+Z_{t}
\end{equation}
\noindent where $\phi_0$ and $\phi_1$ are real numbers, which are referred to as parameters. $Z_t$ is white noise with mean zero and variance $\sigma^2$. \\

\noindent \textbf{Conditional expectation}:
\begin{equation*}
E\left(X_{t} | X_{t-1}\right)=E\left(\phi_{0}+\phi_{1} X_{t-1}+Z_{t} | X_{t-1}\right)=\phi_{0}+\phi_{1} X_{t-1}
\end{equation*}

\noindent Under the stationarity\footnote{necessary and sufficient condition $|\phi| < 1$} condition, we can get
\begin{equation*}
\mu = \phi_0 + \phi_1\mu \longrightarrow E\left(X_{t}\right) = \mu = \frac{\phi_0}{1-\phi_1}
\end{equation*}

\noindent \textbf{Conditional variance}:
\begin{equation*}
\operatorname{Var}\left(X_{t} | X_{t-1}\right)=\sigma^{2}
\end{equation*}

\noindent Under the stationarity condition, we can get
\begin{equation*}
\begin{array}{ll} {\operatorname{Var}\left(X_{t}\right) } & ={\operatorname{Var}\left(\phi_0 + \phi_1X_{t-1} + Z_t\right)} \\ & ={\phi_1^2 \operatorname{Var}\left(X_{t-1}\right) + \operatorname{Var}\left(Z_{t}\right)} \\ & ={\phi_1^2 \operatorname{Var}\left(X_{t-1}\right) + \sigma^2}
\end{array}
\end{equation*}

\noindent so we finally get\footnote{if $\operatorname{Var}\left(X_{t}\right) > 0$, then $1 - \phi_1^2 > 0$, so $\phi_1^2 < 1$.}
\begin{equation*}
\operatorname{Var}\left(X_{t}\right) = \frac{\sigma^2}{1 - \phi_1^2}
\end{equation*}

\noindent \textbf{The ACF of AR(1)}: \\

\noindent We can rewrite the formula \eqref{AR1} as:
\begin{equation*}
X_{t}-\mu=\phi_{1}\left(X_{t-1}-\mu\right)+Z_{t}
\end{equation*}

\noindent If we times $X_t$ in two sides and take expectation, we know $Z_t$ is independent to sequence $\{X_t\}$, then
\begin{equation*}
E\left[X_{t-k}\left(X_{t}-\mu\right)\right]=\phi_{1} E\left[X_{t-k}\left(X_{t-1}-\mu\right)\right]+E\left[Z_tX_{t-k}\right]=\phi_{1} E\left[X_{t-k}\left(X_{t-1}-\mu\right)\right]
\end{equation*}

\noindent We know that
\begin{equation*}
\gamma_k = \operatorname{Cov}\left(X_{t-k}, (X_{t} - \mu)\right) = E\left[X_{t-k}\left(X_{t}-\mu\right)\right]
\end{equation*}

\noindent so we can get
\begin{equation*}
\gamma_k = \phi_{1}\gamma_{k-1}, \forall k>0
\end{equation*}

\noindent so
\begin{equation*}
\gamma_{k}=\phi_1 \gamma_{k-1}=\phi_1^{2} \gamma_{k-2}=\cdots=\phi_1^{k} \gamma_{0}
\end{equation*}

\noindent so
\begin{equation*}
\rho_{k}=\frac{\gamma_{k}}{\gamma_{0}}=\phi_1^{k}
\end{equation*}

\noindent In general, $\rho_k = \phi_1^k$ and ACF $\rho_k$ decays exponentially as k increases. \\

\noindent \textbf{Forecast}: \\
\begin{itemize}
	\item 1-step ahead forecast at time $n$, the forecast origin:
	\begin{equation*}
	\hat{X}_{n}(1) = E\left(X_{n+1}| \mathscr{F}_n\right) = E\left(\phi_{0}+\phi_{1} X_{n} + Z_{n+1}| \mathscr{F}_n\right) = \phi_{0}+\phi_{1} X_{n}
	\end{equation*}
	
	\noindent then 1-step ahead forecast error:
	\begin{equation*}
	e_{n}(1)=X_{n+1}-\hat{X}_{n}(1)=Z_{n+1}
	\end{equation*}
	
	\noindent Variance of 1-step ahead forecast error:
	\begin{equation*}
	\operatorname{Var}\left[e_{n}(1)\right]=\operatorname{Var}\left(Z_{n+1}\right)=\sigma^{2}
	\end{equation*}
	
	\item 2-step ahead forecast:
	\begin{equation*}
	\hat{X}_{n}(2) = E\left(X_{n+2}| \mathscr{F}_n\right) = E\left(\phi_{0}+\phi_{1} X_{n+1} + Z_{n+2}| \mathscr{F}_n\right) = \phi_{0}+\phi_{1} \hat{X}_{n}(1)
	\end{equation*}
	
	\noindent then 1-step ahead forecast error:
	\begin{equation*}
	\begin{array}{ll} {e_{n}(2)} & = {X_{n+2} - \hat{X}_{n}(2)}\\ & = {\phi_{0} + \phi_{1}X_{n+1} + Z_{n+2} - \phi_{0} - \phi_{1} \hat{X}_{n}(1)}\\ & = {\phi_{1}X_{n+1} + Z_{n+2} - \phi_{1} \hat{X}_{n}(1)}\\ & = {\phi_{1}(\phi_{0}+\phi_{1} X_{n}+Z_{n+1}) + Z_{n+2} - \phi_1(\phi_{0}+\phi_{1} X_{n})}\\ & = {Z_{n+2} + \phi_1 Z_{n+1}}
	\end{array}
	\end{equation*}
	
	\noindent Variance of 1-step ahead forecast error:
	\begin{equation*}
	\operatorname{Var}\left[e_{n}(2)\right]=\operatorname{Var}\left(Z_{n+2} + \phi_1 Z_{n+1}\right)=(1+\phi_1^2)\sigma^{2}
	\end{equation*}
	
	\noindent which is greater than or equal to $\operatorname{Var}\left[e_{n}(1)\right]$, implying that uncertainty in forecasts increases as the number of step increases.
\end{itemize}

\noindent \textbf{Compact form}:\\

\noindent Formula \eqref{AR1} also can be written as:
\begin{equation*}
\left(1-\phi_{1} B\right) X_{t}=\phi_{0}+Z_{t}
\end{equation*}
where $B$ is lag operator.

\subsubsection{AR(2) model}
The form of AR(2) can be written as:
\begin{equation}\label{AR2}
X_{t}=\phi_{0}+\phi_{1} X_{t-1}+\phi_{2} X_{t-2}+Z_{t}
\end{equation}

\noindent or
\begin{equation*}
\left(1-\phi_{1} B-\phi_{2} B^{2}\right) X_{t}=\phi_{0}+Z_{t}
\end{equation*}

\noindent \textbf{Staionarity condition}: all roots of $\left(1-\phi_{1} z-\phi_{2} z^{2}\right)=0$ lie outside the unit circle.\\

\noindent If we consider reciprocal of root $z$, we can rewrite the characteristic equation $\left(1-\phi_{1} z-\phi_{2} z^{2}\right)=0$ as
\begin{equation}\label{AR21}
\left(\pi^{2} - \phi_{1} \pi - \phi_{2} \right) = 0
\end{equation}
where the $\pi$ is the reciprocal of $z$. Since $|z| > 1$, we know that $|\pi| < 1$.\\

\noindent Supposed $\pi_1$ and $\pi_2$ are the roots of \eqref{AR21}, then we can get
\begin{equation*}
\left\{\begin{array}{cl}{\pi_1 + \pi_2 } & = {\phi_1}\\
{\pi_1 \pi_2} & = {-\phi_2}
\end{array}\right.
\end{equation*}

\noindent While
\begin{equation*}
\begin{array}{cl}{\phi_1 + \phi_2 - 1} & = {\pi_1 + \pi_2 - \pi_1\pi_2 - 1}\\
& = {(\pi_1-1)(1-\pi_2)} < 0
\end{array}
\end{equation*}

\noindent and
\begin{equation*}
\begin{array}{cl}{\phi_1 - \phi_2 - 1} & = {- \pi_1\pi_2 - \pi_1 - \pi_2 - 1}\\
& = {- (\pi_1 + 1)(\pi_2 + 1)} < 0
\end{array}
\end{equation*}

\noindent so we get
\begin{equation*}
\left\{\begin{array}{lr}{\phi_1 + \phi_2 - 1 < 0}\\ {\phi_1 - \phi_2 - 1 < 0}
\end{array}\right. \longrightarrow \phi_2 < 1 \pm \phi_1
\end{equation*}

\noindent Since $|\pi_1| < 1, |\pi_2| < 1$, we can get $|\phi_2| = |\pi_1\pi_2| < 1$, thus
\begin{equation*}
\left\{\begin{array}{cc}{|\phi_2| < 1}\\
{\phi_2 < 1 \pm \phi_1}	
\end{array}\right.
\end{equation*}

\noindent \textbf{Stochastic business cycle}: As the characteristic equation is
\begin{equation*}\label{key}
1-\phi_1 z - \phi_2 z^2 = 0,
\end{equation*}
the discriminant of root is $\Delta = \phi_1^2 + 4\phi_2$.\\

\noindent If $\phi_{1}^{2}+4 \phi_{2}<0$, the root is complex. The roots can be written as:
\begin{equation*}
z_{1}, z_{2}=\frac{\phi_{1}}{-2 \phi_{2}} \pm i \frac{\sqrt{\left|\phi_{1}^{2}+4 \phi_{2}\right.} |}{-2 \phi_{2}}
\end{equation*}

\noindent then the modulus is
\begin{equation*}
\left|z_{i}\right|=\frac{1}{\sqrt{-\phi_{2}}}
\end{equation*}

\noindent Since $x = |x| (\cos \theta + i \sin \theta)$, then radian is
\begin{equation*}
\theta=\cos ^{-1} \frac{\phi_{1}}{2 \sqrt{-\phi_{2}}}
\end{equation*}

\noindent so the cycle of this radian is
\begin{equation*}
k = \frac{2\pi}{\cos ^{-1} \phi_{1} / (2 \sqrt{-\phi_{2}})}
\end{equation*}

\noindent If we denote the solutions of the polynomial as $a \pm b i$, where $i = \sqrt{-1}$, then
\begin{equation*}
x = a \pm b i = \sqrt{a^2 + b^2} (\cos \theta + i \sin \theta)
\end{equation*}

\noindent so that
\begin{equation*}
k=\frac{2 \pi}{\cos ^{-1}\left(a / \sqrt{a^{2}+b^{2}}\right)}
\end{equation*}

\noindent \textbf{The ACF of AR(2)}:\\

\noindent The \eqref{AR2} can be rewritten as
\begin{equation*}
X_{t} - \mu = \phi_{1} (X_{t-1} - \mu) + \phi_{2} (X_{t-2} - \mu) + Z_{t}
\end{equation*}

\noindent If we times $X_{t-k}$ in two sides and take expectation, then
\begin{equation*}
\begin{array}{ll} {E\left[X_{t-k}\left(X_{t}-\mu\right)\right] } & = {\phi_{1} E\left[X_{t-k}\left(X_{t-1}-\mu\right)\right] + \phi_{2} \left[X_{t-k}(X_{t-2} - \mu)\right] + E\left[Z_tX_{t-k}\right]}\\ & = {\phi_{1} E\left[X_{t-k}\left(X_{t-1}-\mu\right)\right] + \phi_{2} \left[X_{t-k}(X_{t-2} - \mu)\right]}\\ & = {\phi_1 \gamma_{k-1} + \phi_2 \gamma_{k-2}}
\end{array}
\end{equation*}
so that
\begin{equation*}
\rho_{k}=\phi_{1} \rho_{k-1}+\phi_{2} \rho_{k-2}, \quad k \geq 2
\end{equation*}

\noindent \textbf{Forecast}: Similar to AR(1) model.

\subsection{Moving Average (MA) model}
\subsubsection{MA(1) model}
The form of MA(1) model is:
\begin{equation}\label{MA1}
X_t = \mu + Z_t + \theta_1 Z_{t-1}
\end{equation}

\noindent \textbf{Expectation}: 
\begin{equation*}
E(X_t) = \mu
\end{equation*}

\noindent \textbf{Variance}:
\begin{equation*}
\operatorname{Var}\left(r_{t}\right)=\left(1+\theta_1^{2}\right) \sigma^{2}
\end{equation*}

\noindent \textbf{Covariance}:
\begin{equation*}
\gamma_1 = Cov(X_t, X_{t-1}) =  E\left[\left(X_{t}-\mu\right)\left(X_{t-1}-\mu\right)\right] = E\left[\left(Z_{t}+\theta_{1} Z_{t-1}\right)\left(Z_{t-1}+\theta_{1} Z_{t-2}\right)\right]=\theta_{1} E(Z_{t-1}^{2})=\sigma^{2} \theta_{1}
\end{equation*}

\noindent As for $\forall k > 1$, 
\begin{equation*}
\gamma_{k}=E\left[\left(Z_{t}+\theta_{1} Z_{t-1}\right)\left(Z_{t-k}+\theta_{1} Z_{t-k-1}\right)\right]=0
\end{equation*}

\noindent so that
\begin{equation*}
\gamma_{k}=\left\{\begin{array}{ll}{\sigma^{2}\left(1+\theta_{1}^{2}\right),} & {k=0} \\ {\sigma^{2} \theta_{1},} & {k=1} \\ {0,} & {k>1}\end{array}\right.
\end{equation*}

\noindent so the ACF of MA(1) model is
\begin{equation*}
\rho_{k}=\left\{\begin{array}{ll}{1,} & {k=0} \\ {\frac{\theta_{1}}{1+\theta_{1}^{2}},} & {k=1} \\ {0,} & {k>1}\end{array}\right.
\end{equation*}

\noindent \textbf{Forecast}:
\begin{itemize}
	\item 1-step ahead: (at origin $t=n$)
	\begin{equation*}
	\hat{X}_{n}(1) = E\left(X_{n+1} | X_{1}, \ldots, X_{n}\right) = \mu + \theta_{1} Z_{n}
	\end{equation*}
	
	1-step ahead forecast error:
	\begin{equation*}
	e_n(1) = Z_{n+1} 
	\end{equation*}
	with variance $\sigma^2$.
	
	\item multi-step ahead:
	\begin{equation*}
	\hat{X}_{n}(k) = \mu
	\end{equation*}
	for $k > 1$.
\end{itemize}

\noindent \textbf{Invertibility}:
\begin{itemize}
	\item Concept: $X_t$ is a proper linear combination of $Z_t$ and the past observations $\left\{X_{t-1}, X_{t-2}, \cdots\right\}$.
	
	\item For an invertible model, the dependence of $X_t$ on $X_{t-k}$ converges to zero as $k$ increases.
	
	\item Condition: $|\theta| < 1$
	
	\item Invertibility of MA models is the dual property of stationarity for AR models.
\end{itemize}

\subsubsection{MA(2) model}
The form of MA(2) can be written as:
\begin{equation*}
X_t = \mu + Z_t + \theta_1 Z_{t-1} + \theta_2 Z_{t-2}
\end{equation*}
or
\begin{equation*}
X_t - \mu = (1 + \theta_1 B + \theta_2 B^2) Z_t
\end{equation*}

\noindent \textbf{Invertibility}: all the roots of $\theta(z)=1+\theta_{1} z+\theta_{2} z^{2}=0$ lie outside the unit circle.

\subsection{ARMA(p,q)}
The ARMA(p,q) model can be represented as:
\begin{equation}\label{ARMA1}
X_{t}=\phi_{0}+\phi_{1} X_{t-1}+\cdots+\phi_{p} X_{t-p}+Z_{t}+\theta_{1} Z_{t-1}+\cdots+\theta_{q} Z_{t-q}
\end{equation}
where $Z_t \sim N(0, \sigma^2)$.

\section{Seasonal time series}
\subsection{Multiplicative model}
The form can be represented as:
\begin{equation*}
X_{t}-X_{t-1}-X_{t-4}+X_{t-5}=Z_{t}-\theta_{1} Z_{t-1}-\theta_{4} Z_{t-4}+\theta_{1} \theta_{4} Z_{t-5}
\end{equation*}
or
\begin{equation*}
(1-B)\left(1-B^{4}\right) X_{t}=\left(1-\theta_{1} B\right)\left(1-\theta_{4}B^4\right) Z_{t}
\end{equation*}

\noindent Supposed that
\begin{equation*}
Y_{t}=(1-\theta_1 B)\left(1-\theta_4 B^{4}\right) Z_{t}=Z_{t}-\theta_1 Z_{t-1}-\theta_4 Z_{t-4} + \theta_1 \theta_4 Z_{t-4-1}
\end{equation*}

\noindent then
\begin{itemize}
	\item $E (Y_t) = 0$
	\item $\gamma_{0}=\operatorname{Var}\left(Y_{t}\right)=\left(1+\theta_1^{2}\right)\left(1+\theta_4^{2}\right) \sigma^{2}$
	\item $\gamma_{1}=-\theta_1\left(1+\theta_4^{2}\right) \sigma^{2}$
	\item $\gamma_{3}=\theta_1 \theta_4 \sigma^{2}$
	\item $\gamma_{4}=-\theta_4\left(1+\theta_1^{2}\right) \sigma^{2}$
	\item $\gamma_{5}=\theta_1 \theta_4 \sigma^{2}$
	\item $\gamma_{k}=0, k \neq 0, 1, 3, 4, 5$
\end{itemize}

\noindent so the ACF of $\{Y_t\}$ is
\begin{itemize}
	\item $\rho_{1}=\frac{-\theta_1}{1+\theta_1^{2}}$
	\item $\rho_{4}=\frac{-\theta_4}{1+\theta_4^{2}}$
	\item $\rho_{3}=\rho_{5}=\rho_{1} \rho_{4}=\frac{\theta_1 \theta_4}{\left(1+\theta_1^{2}\right)\left(1+\theta_4^{2}\right)}$
	\item $\rho_k = 0$, for other $k$
\end{itemize}

\noindent As for 
\begin{equation*}
(1-B)\left(1-B^{s}\right) X_{t}=(1-\theta B)\left(1-\Theta B^{s}\right) Z_{t}
\end{equation*}
we can get covariance as
\begin{itemize}
	\item $E (Y_t) = 0$
	\item $\gamma_{0}=\operatorname{Var}\left(Y_{t}\right)=\left(1+\theta^{2}\right)\left(1+\Theta^{2}\right) \sigma^{2}$
	\item $\gamma_{1}=-\theta\left(1+\Theta^{2}\right) \sigma^{2}$
	\item $\gamma_{s-1}=\theta \Theta \sigma^{2}$
	\item $\gamma_{s}=-\Theta\left(1+\theta^{2}\right) \sigma^{2}$
	\item $\gamma_{s+1}=\theta \Theta \sigma^{2}$
	\item $\gamma_{k}=0, k \neq 0, 1, s-1, s, s+1$
\end{itemize}

\noindent so the ACF of $\{Y_t\}$ becomes
\begin{itemize}
	\item $\rho_{1}=\frac{-\theta}{1+\theta^{2}}$
	\item $\rho_{s}=\frac{-\Theta}{1+\Theta^{2}}$
	\item $\rho_{s-1}=\rho_{s+1}=\rho_{1} \rho_{s}=\frac{\theta \Theta}{\left(1+\theta^{2}\right)\left(1+\Theta^{2}\right)}$
	\item $\rho_k = 0$, for other $k$
\end{itemize}

\section{GARCH models - Condtional heteroscedastic models}
\subsection{ARCH effect}
Let $\xi_{t}=Z_{t}^{2}-E Z_{t}^{2}$. If there is not ARCH effect, then the ACF of $\xi_t$ are
all zero.

\noindent In 1983, McLeod and Li use Ljung-Box to test the null $H_0$: the ACF $\rho_k$ of $\xi_t$ are all zero, i.e,
\begin{equation*}
H_0: \rho_{1}=\cdots=\rho_{m}=0
\end{equation*}
Let $\hat{\rho}_{k}$ be the sample ACF of $\{\xi_t\}$. We use the Ljung-Box:
\begin{equation*}
Q(m)=n(n-1) \sum_{k=1}^{m} \frac{\hat{\rho}_{k}^{2}}{n-k} \sim \chi^{2}(m)
\end{equation*}
If test statistics is significant, then reject $H_0$ and there exists ARCH effect; otherwise, we can not reject the null hypothesis, which means there do not exist ARCH effect.\\

\noindent In 1982, Engle use (Lagrange multiplier) LM test to test ARCH effect:
Assumed that
\begin{equation*}
\xi_{t}=\alpha_{1} \xi_{t-1}+\cdots+\alpha_{m} \xi_{t-m}+e_{t}=\beta^{\prime} U_{t}+e_{t}
\end{equation*}
where $\beta=\left(\alpha_{1}, \cdots, \alpha_{m}\right)^{\prime}$ and $X_{t}=\left(\xi_{t-1}, \cdots, \xi_{t-m}\right)$ and use LM test for the null:
\begin{equation*}
H_{0} : \alpha_{1}=\cdots=\alpha_{m}=0
\end{equation*}
The LM test statistics:
\begin{equation*}
L M=\left(\sum_{t=1}^{n} X_{t}^{\prime}\right)\left[\sum_{t=1}^{n} X_{t} X_{t}^{\prime}\right]^{-1}\left(\sum_{t=1}^{n} X_{t}\right) \sim \chi^{2}(m)
\end{equation*}

\subsection{ARCH model}
\begin{equation*}
Z_{t}=\sigma_{t} \epsilon_{t}, \quad \sigma_{t}^{2}=\alpha_{0}+\alpha_{1} Z_{t-1}^{2}+\cdots+\alpha_{m} Z_{t-m}^{2}
\end{equation*}
where $\{\epsilon_{t}\}$ is a sequence of iid r.v. with mean 0 and variance 1, also $\alpha_0 > 0$ and $\alpha_i \geq 0$ for $i > 0$.

\subsection{GARCH model}
\begin{equation*}
Z_{t} =\sigma_{t} \epsilon_{t}, \quad \sigma_{t}^{2} =\alpha_{0}+\sum_{i=1}^{m} \alpha_{i} Z_{t-i}^{2}+\sum_{i=1}^{s} \beta_{i} \sigma_{t-i}^{2}
\end{equation*}
where $\{\epsilon_{t}\}$ is defined above, also $\alpha_0 > 0$, $\alpha_i \geq 0$ and $\beta_j \geq 0$. \\

\noindent Focus on a GARCH(1,1) model:
\begin{equation*}
Z_{t} =\sigma_{t} \epsilon_{t}, \quad \sigma_{t}^{2}=\alpha_{0}+\alpha_{1} Z_{t-1}^{2}+\beta_{1} \sigma_{t-1}^{2}
\end{equation*}

\noindent In order to calculate expectation $E(Z_t)$, then firstly calculate conditional expectaion:
\begin{equation*}
\begin{aligned} E\left(Z_{t} | F_{t-1}\right)=E\left(\sigma_{t} \epsilon_{t} | F_{t-1}\right)&=E\left(\sqrt{\alpha_{0}+\alpha_{1} Z_{t-1}^{2}+\beta_{1}\sigma_{t-1}^{2}} \cdot \epsilon_{t} | F_{t-1}\right)\\ &=\sqrt{\alpha_{0}+\alpha_{1} Z_{t-1}^{2}+\beta_{1}\sigma_{t-1}^{2}} \cdot E\left(\epsilon_{t} | F_{t-1}\right)\\ &= \sigma_{t} E\left(\epsilon_{t} | F_{t-1}\right)=0
\end{aligned}
\end{equation*}
so that
\begin{equation*}
E (Z_{t})=E\left[E\left(Z_{t} | F_{t-1}\right)\right]=0
\end{equation*}

\noindent Next, we try to calculate the variance of $Z_t$. Supposed that sequence $\{Z_t\}$ exists strictly stationary solution, then
\begin{equation*}
\begin{aligned} \operatorname{Var}\left(Z_{t}\right) &=E\left(Z_{t}^{2}\right) \\ &=E\left[E\left(Z_{t}^{2} | F_{t-1}\right)\right] \\ &=E\left[E\left(\sigma_{t}^{2} \epsilon_{t}^{2} | F_{t-1}\right)\right] \\ &=E\left[\sigma_{t}^{2} E\left(\epsilon_{t}^{2} | F_{t-1}\right)\right] \\ &=E\left[\sigma_{t}^{2} E\left(\epsilon_{t}^{2}\right)\right] \\ &=E\left[\sigma_{t}^{2}\right] \\ &=E\left[\alpha_{0}+\alpha_{1} Z_{t-1}^{2}+\beta_{1} \sigma_{t-1}^{2}\right] \\ &= \alpha_{0}+\alpha_{1} E\left(Z_{t-1}^{2}\right)+\beta_{1} E\left(\sigma_{t-1}^{2}\right)\\ &=\alpha_{0}+\alpha_{1} E\left(Z_{t-1}^{2}\right)+\beta_{1} E\left[E\left(Z_{t-1}^{2} | F_{t-2}\right)\right] \\ &=\alpha_{0}+\left(\alpha_{1}+\beta_{1}\right) E\left(Z_{t-1}^{2}\right) \end{aligned}
\end{equation*}
Let $E (Z_{t}^{2}) = E (Z_{t-1}^{2})$, then we get
\begin{equation*}
\operatorname{Var}\left(Z_{t}\right)=E (Z_{t}^{2})=\frac{\alpha_{0}}{1-\alpha_{1}-\beta_{1}}
\end{equation*}

\subsubsection{Forecast}
Since
\begin{equation*}
\sigma_{t+1}^{2}=\alpha_{0}+\alpha_{1} Z_{t}^{2}+\beta_{1} \sigma_{t}^{2} \in F_{t}
\end{equation*}
then 1-step ahead forescast:
\begin{equation*}
\sigma_{t}^{2}(1)=E\left(\sigma_{t+1}^{2} | F_{t}\right)=\sigma_{t+1}^{2}=\alpha_{0}+\alpha_{1} Z_{t}^{2}+\beta_{1} \sigma_{t}^{2}
\end{equation*}

\noindent As for 2-step ahead forecast, since
\begin{equation*}
Z_{t}^{2}=\sigma_{t}^{2} \epsilon_{t}^{2}
\end{equation*}
and
\begin{equation*}
\begin{aligned} \sigma_{t+2}^{2} &=\alpha_{0}+\alpha_{1} Z_{t+1}^{2}+\beta_{1} \sigma_{t+1}^{2} \\ &=\alpha_{0}+\alpha_{1} \sigma_{t+1}^{2} \epsilon_{t+1}^{2}+\beta_{1} \sigma_{t+1}^{2} \\ &=\alpha_{0}+\left(\alpha_{1} \epsilon_{t+1}^{2}+\beta_{1}\right) \sigma_{t+1}^{2} \end{aligned}
\end{equation*}
then
\begin{equation*}
\sigma_{t}^{2}(2)=E\left(\sigma_{t+2}^{2} | F_{t}\right)=\alpha_{0}+E\left(\alpha_{1} \epsilon_{t+1}^{2}+\beta_{1} | F_{t}\right) \sigma_{t+1}^{2}=\alpha_{0}+\left(\alpha_{1}+\beta_{1}\right) \sigma_{t}^{2}(1)
\end{equation*}
In general, since
\begin{equation*}
\sigma_{t+\ell}^{2}=\alpha_{0}+\alpha_{1} \varepsilon_{t+\ell-1}^{2} \sigma_{t+\ell-1}^{2}+\beta_{1} \sigma_{t+\ell-1}^{2}=\alpha_{0}+\left(\alpha_{1} \varepsilon_{t+\ell-1}^{2}+\beta_{1}\right) \sigma_{t+\ell-1}^{2}
\end{equation*}
we have
\begin{equation*}
\begin{aligned} \sigma_{t}^{2}(\ell) &=E\left\{\sigma_{t+\ell}^{2} | F_{t}\right\}=\alpha_{0}+E\left\{\left(\alpha_{1} \epsilon_{t+\ell-1}^{2}+\beta_{1}\right) \sigma_{t+\ell-1}^{2} | F_{t}\right\} \\ &=\alpha_{0}+E\left\{E\left[\left(\alpha_{1} \epsilon_{t+\ell-1}^{2}+\beta_{1}\right) \sigma_{t+\ell-1}^{2} | F_{t+\ell-2}\right] | F_{t}\right\}\\
&=\alpha_{0}+E\left\{\sigma_{t+\ell-1}^{2} E\left[\alpha_{1} \varepsilon_{t+\ell-\ell-1}^{2}+\beta_{1} | F_{t+\ell-2}\right] | F_{t}\right\} \\
&=\alpha_{0}+\left\{\sigma_{t+\ell-1}^{2}\left(\alpha_{1}+\beta_{1}\right) | F_{t}\right\} \\ &=\alpha_{0}+\left(\alpha_{1}+\beta_{1}\right) \sigma_{t}^{2}(\ell-1)
\end{aligned}
\end{equation*}

\subsection{IGARCH model}
An IGARCH model is as follows:
\begin{equation*}
Z_{t}=\sigma_{t} \epsilon_{t}, \sigma_{t}^{2}=\alpha_{0}+\beta_{1} \sigma_{t-1}^{2}+\left(1-\beta_{1}\right) Z_{t-1}^{2}
\end{equation*}
In this case,
\begin{equation*}
\sigma_{t}^{2}(\ell)=\sigma_{t}^{2}(1)+(\ell-1) \alpha_{0}
\end{equation*}
where $h$ is the forecast origin. The effect of $\sigma_{t}^{2}(\ell)$ on future is persistent, and the volatility forecasts form a straight line with slope $\alpha_{0}$.

\subsection{GARCH-M model}
\begin{equation*}
\begin{array}{c}{r_{t}=\mu+c \sigma_{t}^{2}+Z_{t}} \\ {Z_{t}=\sigma_{t} \epsilon_{t}, \sigma_{t}^{2}=\alpha_{0}+\alpha_{1} Z_{t-1}^{2}+\beta_{1} \sigma_{t-1}^{2}}\end{array}
\end{equation*}
where $c$ is referred to as \textbf{risk premium}, which is expected to be positive.

\subsection{EGARCH model}
Asymmetry in responses to + and - returns:
\begin{equation*}
g\left(\epsilon_{t}\right)=\theta \epsilon_{t}+\left[\left|\epsilon_{t}\right|-E\left(\left|\epsilon_{t}\right|\right)\right]
\end{equation*}
with $E\left[g\left(\epsilon_{t}\right)\right]=0$. To see asymmetry of $g\left(\epsilon_{t}\right)$, rewrite it as
\begin{equation*}
g\left(\epsilon_{t}\right)=\left\{\begin{array}{l}{(\theta+1) \epsilon_{t}-E\left(\left|\epsilon_{t}\right|\right) \text { if } \epsilon_{t} \geq 0} \\ {(\theta-1) \epsilon_{t}-E\left(\left|\epsilon_{t}\right|\right) \text { if } \epsilon_{t}<0}\end{array}\right.
\end{equation*}

\noindent An EGARCH(m,s) model is formulated as:
\begin{equation*}
Z_{t}=\sigma_{t} \epsilon_{t}, \ln \left(\sigma_{t}^{2}\right)=\alpha_{0}+\sum_{i=1}^{m} \alpha_{i} g\left(\epsilon_{t-i}\right)+\sum_{i=1}^{s} \beta_{i} \ln \left(\sigma_{t-i}^{2}\right)
\end{equation*}

\noindent Consider an EGARCH(1,1) model
\begin{equation*}
Z_{t}=\sigma_{t} \epsilon_{t}, \quad(1-\beta B) \ln \left(\sigma_{t}^{2}\right)=\alpha_{0}+\alpha g\left(\epsilon_{t-1}\right)
\end{equation*}
Under normality, $E\left(\left|\epsilon_{t}\right|\right)=\sqrt{2 / \pi}$ and the model becomes
\begin{equation*}
(1-\beta B) \ln \left(\sigma_{t}^{2}\right)=\left\{\begin{array}{l}{\alpha_{*}+\alpha(\theta+1) \epsilon_{t-1} \text { if } \epsilon_{t-1} \geq 0} \\ {\alpha_{*}+\alpha(\theta-1) \epsilon_{t-1} \text { if } \epsilon_{t-1}<0}\end{array}\right.
\end{equation*}
where $\alpha_{*}=\alpha_{0}-\alpha \sqrt{2 / \pi}$.

\subsection{TGARCH model}
The Threshold GARCH (TGARCH) or GJR Model A TGARCH(s,m) or GJR(s,m) model is defined as
\begin{equation*}
\begin{array}{c}{r_{t}=\mu_{t}+Z_{t}, \quad Z_{t}=\sigma_{t} \epsilon_{t}} \\ {\sigma_{t}^{2}=\alpha_{0}+\sum_{i=1}^{s}\left(\alpha_{i}+\gamma_{i} N_{t-i}\right) Z_{t-i}^{2}+\sum_{j=1}^{m} \beta_{j} \sigma_{t-j}^{2}}\end{array}
\end{equation*}
where $N_{t-i}$ is an indicator variable such that
\begin{equation*}
N_{t-i}=1 \text { if } Z_{t-i}<0, \text { and } = 0 \text{ otherwise }
\end{equation*}

\section{Multivariate time series models}
The real data $Z_t$ is a vector:
\begin{equation*}
Z_{t}=\left[ \begin{array}{c}{Z_{1 t}} \\ {Z_{2 t}} \\ {\vdots} \\ {Z_{m t}}\end{array}\right]
\end{equation*}
$Z_t$ is called an $m$-dimensional vector time series.

\begin{itemize}
	\item Mean
	\begin{equation*}
	E Z_{t}=\left[ \begin{array}{c}{E Z_{1 t}} \\ {E Z_{2 t}} \\ {\vdots} \\ {E Z_{m t}}\end{array}\right]=\left[ \begin{array}{c}{\mu_{1 t}} \\ {\mu_{2 t}} \\ {\vdots} \\ {\mu_{m t}}\end{array}\right] \equiv \mu
	\end{equation*}
	\item Covariance
	\begin{equation*}
	\begin{aligned}
	\Gamma(k)&=\operatorname{Cov}\left(Z_{t}, Z_{t+k}\right)=E\left[\left(Z_{t}-\mu\right)\left(Z_{t+k}-\mu\right)^{\prime}\right]\\
	&=\left[ \begin{array}{cccc}{\gamma_{11}(k)} & {\gamma_{12}(k)} & {\cdots} & {\gamma_{1 m}(k)} \\ {\gamma_{21}(k)} & {\gamma_{22}(k)} & {\cdots} & {\gamma_{2 m}(k)} \\ {\vdots} & {\vdots} & {\vdots} & {\vdots} \\ {\gamma_{m 1}(k)} & {\gamma_{m 2}(k)} & {\cdots} & {\gamma_{m m}(k)}\end{array}\right]
	\end{aligned}
	\end{equation*}
	\item Corralation matrix
	\begin{equation*}
	\rho_{0}=\left(\rho_{i j}(0)\right)_{k \times k}=D^{-1} \Gamma_{0} D^{-1}
	\end{equation*}
	where $D=\operatorname{diag}\left(\sqrt{\Gamma_{11}(0)}, \ldots, \sqrt{\Gamma_{k k}(0) )}\right.$
\end{itemize}

\noindent As for $k$-dimension time series,
\begin{equation*}
\begin{aligned} \Gamma_{i j}(l) &=\operatorname{Cov}\left(r_{i t}, r_{j, t-l}\right)=\operatorname{Cov}\left(r_{j, t-l}, r_{i, t}\right)=\operatorname{Cov}\left(r_{j, t}, r_{i, t+l}\right) \\ &=\operatorname{Cov}\left(r_{j, t}, r_{i, t-(-l)}\right)=\Gamma_{j i}(-l) \end{aligned}
\end{equation*}
so that
\begin{equation*}
\Gamma_{-l}=\Gamma_{l}^{T}
\end{equation*}
and also we can get
\begin{equation*}
\rho_{-l}=\rho_{l}^{T}
\end{equation*}

\subsection{The vector AR(1) model}
\begin{itemize}
	\item Model
	\begin{equation*}
	\left(I-\Phi_{1} B\right) \dot{X}_{t}=Z_{t} \text{,\quad or} \dot{X}_{t}=\Phi_{1} \dot{X}_{t-1}+Z_{t}
	\end{equation*}
	when $m = 2$,
	\begin{equation*}
	\left[ \begin{array}{c}{\dot{X}_{1 t}} \\ {\dot{X}_{2 t}}\end{array}\right]=\left[ \begin{array}{cc}{\phi_{11}} & {\phi_{12}} \\ {\phi_{21}} & {\phi_{22}}\end{array}\right] \left[ \begin{array}{c}{\dot{X}_{1, t-1}} \\ {\dot{X}_{2, t-1}}\end{array}\right]+\left[ \begin{array}{c}{Z_{1 t}} \\ {Z_{2 t}}\end{array}\right]
	\end{equation*}
	
	\item Stationarity conditions\\
	
	\noindent All the roots of the determinant $\left|I-\Phi_{1} z\right|=0$ lie outside the unit cricle.\\
	
	\noindent In this case,
	\begin{equation*}
	\left(I-\Phi_{1} B\right)^{-1}=I+\Phi_{1} B+\Phi_{1}^{2} B^{2}+\cdots
	\end{equation*}
	and
	\begin{equation*}
	\dot{X}_{t}=Z_{t}+\Phi_{1} Z_{t-1}+\Phi_{1}^{2} Z_{t-2}+\cdots
	\end{equation*}
	
	\item Covariance matrix function
	\begin{equation*}
	\Gamma(k)=\left\{\begin{array}{ll}{\Gamma(-1) \Phi_{1}^{\prime}+\Sigma,} & {\text { if } k=0} \\ {\Gamma(k-1) \Phi_{1}^{\prime}=\Gamma(0)\left(\Phi_{1}^{\prime}\right)^{k},} & {\text { if } k \geq 1}\end{array}\right.
	\end{equation*}
	In general, $\Gamma(0) \neq I$
	\begin{equation*}
	\Phi_{1}=\Gamma^{\prime}(1) \Gamma^{-1}(0) \quad \text { and } \Sigma=\Gamma(0)-\Gamma^{\prime}(1) \Phi_{1}^{\prime}
	\end{equation*}
	Thus,
	\begin{equation*}
	\Sigma=\Gamma(0)-\Phi_{1} \Gamma(0) \Phi_{1}^{\prime}
	\end{equation*}
	We can find $\Gamma(0)$ by solving the above equation.
\end{itemize}

\subsection{The vector MA(1)model}
\begin{itemize}
	\item Model
	\begin{equation*}
	\dot{Z}_{t}=\left(I-\Theta_{1} B\right) a_{t}, \quad \text { or } \dot{Z}_{t}=a_{t}-\Theta_{1} a_{t-1}
	\end{equation*}
	when $m = 2$,
	\begin{equation*}
	\left[ \begin{array}{c}{\dot{Z}_{1 t}} \\ {\dot{Z}_{2 t}}\end{array}\right]=\left[ \begin{array}{c}{a_{1 t}} \\ {a_{2 t}}\end{array}\right]-\left[ \begin{array}{cc}{\Theta_{11}} & {\Theta_{12}} \\ {\Theta_{21}} & {\Theta_{22}}\end{array}\right] \left[ \begin{array}{c}{a_{1, t-1}} \\ {a_{2, t-1}}\end{array}\right]
	\end{equation*}
	
	\item Invertibility conditions\\
	
	\noindent All the roots of the determinant $\left|I-\Theta_{1} z\right|=0$ lie outside the unit circle.\\
	
	\noindent In this case,
	\begin{equation*}
	\left(I-\Theta_{1} B\right)^{-1}=I+\Theta_{1} B+\Theta_{1}^{2} B^{2}+\cdots
	\end{equation*}
	and
	\begin{equation*}
	Z_{t}=\dot{X}+\Theta_{1} \dot{X}_{t-1}+\Theta_{1}^{2} \dot{X}_{t-2}+\cdots
	\end{equation*}
	
	\item Covariance matrix function
	\begin{equation*}
	\Gamma(0)=\Sigma+\Theta_{1} \Sigma \Theta_{1}^{\prime};
	\end{equation*}
	\begin{equation*}
	\Gamma(k)=\left\{\begin{array}{ll}{-\Sigma \Theta_{1}^{\prime},} & {\text { if } k=1} \\ {-\Theta_{1} \Sigma,} & {\text { if } k=-1} \\ {0,} & {\text { if }|k|>1}\end{array}\right.
	\end{equation*}
\end{itemize}

\section{Partially non-stationary vector time series}
\subsection{Cointegration}
\textbf{Definition} 1. (Engle and Granger 1987): For a series yt with no deterministic components, let
\begin{equation*}
(1-B)^{d} y_{t}=z_{t}
\end{equation*}
If $z_t$ is a stationary ARMA process, we said that $y_t \sim I(d)$.\\

\noindent \textbf{Definition} 1. (Engle and Granger 1987): If all elements of the vector $\mathbf{y}_t$ are $I(d)$ and there exists a vector $\beta \neq 0$ such that
\begin{equation*}
\beta^{\prime} \mathbf{y}_{t} \sim I(d-b)
\end{equation*}
for some $b > 0$, the vector process is said to be cointegrated $CI(d,b)$ and $\beta$ is called cointegrating vector.\\

\noindent \textbf{Example.} The bivariate system:
\begin{equation*}
\begin{array}{l}{y_{1 t}=\gamma y_{2 t}+\varepsilon_{1 t}} \\ {y_{2 t}=y_{2\{t-1\}}+\varepsilon_{2 t}}\end{array}
\end{equation*}
where $\gamma \neq 0$, $\varepsilon_{1 t}$ and $\varepsilon_{2 t}$ being uncorrelated white noise processes.\\

\noindent $\triangle y_{2 t}=\varepsilon_{2 t}$, where $\triangle \equiv 1-B$.

\begin{equation*}
\triangle y_{1 t}=\gamma \triangle y_{2 t}+\triangle \varepsilon_{1 t}=\gamma \varepsilon_{2 t}+\varepsilon_{1 t}-\varepsilon_{1, t-1}
\end{equation*}

\noindent Thus, both $y_{1 t}$ and $y_{2 t}$ are $I(1)$ processes, but the linear combination $y_{1 t}-\gamma y_{2 t}$ is stationary.\\

\noindent Hence $\mathbf{y}_{t}=\left(y_{1 t}, y_{2 t}\right)^{\prime}$ is cointegrated with a cointegrating vector $\beta = (1, -\gamma)^{\prime}$.\\

\noindent In general, if the vector process $\mathbf{y}_{t}$ has $m$ components, then there can be
more than one cointegrating vector $\beta^{\prime}$.











\end{document}